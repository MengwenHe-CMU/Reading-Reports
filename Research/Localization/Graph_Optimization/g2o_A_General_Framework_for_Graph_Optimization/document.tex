\documentclass[letterpaper,10pt]{article}
\usepackage[margin=2cm]{geometry}

\usepackage{graphicx}
\usepackage{amsmath}
\usepackage{amsfonts}
\usepackage{amssymb}
\usepackage[colorlinks]{hyperref}

\title{\textbf{g2o: A General Framework for Graph Optimization}}
\author{Rainer KummerleAuthors, Giorgio Grisetti, Hauke Strasdat, Kurt Konolige, Wolfram Burgard}

\begin{document}
	
\maketitle

\tableofcontents

\begin{center}\rule{\textwidth}{1pt}\end{center}
\section{Resources}

\begin{itemize}
	\item \href{./paper.pdf}{Paper}
	\item \href{./beamer.pdf}{Beamer}
\end{itemize}

\begin{center}\rule{\textwidth}{1pt}\end{center}
\section{Basic Information}

\subsection{Authors}
R. Kummerle, G. Grisetti, and W. Burgard are with the University
of Freiburg. G. Grisetti is also with Sapienza, University of Rome. H. Strasdat is with the Department of Computing, Imperial College London. K. Konolige is with Willow Garage and a Consulting Professor at Stanford University.

\subsection{Conference}

2011 IEEE International Conference on Robotics and Automation (ICRA 2011, Shanghai)

\subsection{Abstract}
\begin{itemize}
	\item Simultaneous Localization And Mapping (SLAM) or Bundle Adjustment (BA) can be phrased as \textbf{least squares optimization} of an error function that can be represented by a \textbf{graph}.
	\item $g^2o$\footnote{\url{https://github.com/MengwenHe-CMU/g2o}} is an open-source C++ framework for optimization graph-based nonlinear error functions.
\end{itemize}

\subsection{Keywords}


\begin{center}\rule{\textwidth}{1pt}\end{center}
\section{Introduction}

\subsection{Problem \& Solution}


\subsection{Objective}


\subsection{Formulation}


\subsection{Contributions}
\begin{itemize}
	\item 
\end{itemize}

\begin{center}\rule{\textwidth}{1pt}\end{center}
\section{Related Work}


\begin{center}\rule{\textwidth}{1pt}\end{center}
\section{Method Description}


\begin{center}\rule{\textwidth}{1pt}\end{center}
\section{Experiment Evaluation}


\begin{center}\rule{\textwidth}{1pt}\end{center}
\section{Conclusion}


\begin{center}\rule{\textwidth}{1pt}\end{center}
\section{Note}

\end{document}